% generated by GAPDoc2LaTeX from XML source (Frank Luebeck)
\documentclass[11pt]{report}
\usepackage{a4wide}
\sloppy \pagestyle{myheadings}
\usepackage{amssymb}
\usepackage[latin1]{inputenc}
\usepackage{makeidx}
\makeindex
\usepackage{color}
\definecolor{DarkOlive}{rgb}{0.1047,0.2412,0.0064}
\definecolor{FireBrick}{rgb}{0.5812,0.0074,0.0083}
\definecolor{RoyalBlue}{rgb}{0.0236,0.0894,0.6179}
\definecolor{RoyalGreen}{rgb}{0.0236,0.6179,0.0894}
\definecolor{RoyalRed}{rgb}{0.6179,0.0236,0.0894}
\definecolor{LightBlue}{rgb}{0.8544,0.9511,1.0000}
\definecolor{Black}{rgb}{0.0,0.0,0.0}
\definecolor{FuncColor}{rgb}{1.0,0.0,0.0}
%% strange name because of pdflatex bug:
\definecolor{Chapter }{rgb}{0.0,0.0,1.0}

\usepackage{fancyvrb}

\usepackage{pslatex}

\usepackage[
        a4paper=true,bookmarks=false,pdftitle={Written with GAPDoc},
        pdfcreator={LaTeX with hyperref package / GAPDoc},
        colorlinks=true,backref=page,breaklinks=true,linkcolor=RoyalBlue,
        citecolor=RoyalGreen,filecolor=RoyalRed,
        urlcolor=RoyalRed,pagecolor=RoyalBlue]{hyperref}

% write page numbers to a .pnr log file for online help
\newwrite\pagenrlog
\immediate\openout\pagenrlog =\jobname.pnr
\immediate\write\pagenrlog{PAGENRS := [}
\newcommand{\logpage}[1]{\protect\write\pagenrlog{#1, \thepage,}}
\newcommand{\Q}{\mathbb{Q}}
\newcommand{\R}{\mathbb{R}}
\newcommand{\C}{\mathbb{C}}
\newcommand{\Z}{\mathbb{Z}}
\newcommand{\N}{\mathbb{N}}
\newcommand{\F}{\mathbb{F}}

\newcommand{\GAP}{\textsf{GAP}}

\newsavebox{\backslashbox}
\sbox{\backslashbox}{\texttt{\symbol{92}}}
\newcommand{\bs}{\usebox{\backslashbox}}

\begin{document}

\logpage{[ 0, 0, 0 ]}
\begin{titlepage}
\begin{center}{\Huge \textbf{AutomataGroups}}\\[1cm]
\hypersetup{pdftitle=AutomataGroups}
\markright{\scriptsize \mbox{}\hfill AutomataGroups \hfill\mbox{}}
{\Large \textbf{Package for computations in groups generated by finite automata}}\\[1cm]
{Version 0.1}\\[1cm]
{December 2006}\\[1cm]
\mbox{}\\[2cm]
{\large \textbf{Yevgen Muntyan  }}\\
{\large \textbf{Dmytro Savchuk   }}\\
\hypersetup{pdfauthor=Yevgen Muntyan  ; Dmytro Savchuk   }
\end{center}\vfill

\mbox{}\\
{\mbox{}\\
\small \noindent \textbf{Yevgen Muntyan  } --- Email: \href{mailto://muntyan@math.tamu.edu}{\texttt{muntyan@\
math.tamu.edu}}}\\
{\mbox{}\\
\small \noindent \textbf{Dmytro Savchuk   } --- Email: \href{mailto://savchuk@math.tamu.edu}{\texttt{savchuk\
@math.tamu.edu}}\\
 --- Homepage: \href{http://www.math.tamu.edu/~savchuk}{\texttt{http://www.math.tamu.edu/\~{}savchuk}}}\\

\noindent \textbf{Address: }\begin{minipage}[t]{8cm}\noindent
 Department of Mathematics\\
 Texas A\&M University\\
 College Station, 77843\\
 TX, USA \end{minipage}
\end{titlepage}

\newpage\setcounter{page}{2}
{\small
\section*{Abstract}
\logpage{[ 0, 0, 1 ]}
This is a manual for an
\begin{verbatim}  AutomataGroups
\end{verbatim}
 package, implementing in \textsf{GAP} basic functions and algorithms for groups generated by finite automat\
a,
contracting groups. }\\[1cm]
{\small
\section*{Copyright}
\logpage{[ 0, 0, 2 ]}
{\copyright} 2006 by Yevgen Muntyan and Dmytro Savchuk }\\[1cm]
{\small
\section*{Acknowledgements}
\logpage{[ 0, 0, 3 ]}
The development of this package was partially supported by National Science
Foundation }\\[1cm]
{\small
\section*{Colophon}
\logpage{[ 0, 0, 4 ]}
The project was originally started in 2000 mostly for personal use. It was
gradually expadnig during consequent years, including both addition of new
algorithms and simplification of user interface. It was used in the process of
classification of groups generated by $3$-state automata over 2-letter
alphabet. }\\[1cm]
\newpage

\def\contentsname{Contents\logpage{[ 0, 0, 5 ]}}

\tableofcontents
\newpage


\chapter{\textcolor{Chapter }{Introduction}}\label{ch:intro}
\logpage{[ 1, 0, 0 ]}
{

\section{\textcolor{Chapter }{Mathematical Background}}\label{sec:math_background}
\logpage{[ 1, 1, 0 ]}
{
 }


\section{\textcolor{Chapter }{Notations and Agreements}}\logpage{[ 1, 2, 0 ]}
{
 }


\section{\textcolor{Chapter }{Quick Example}}\logpage{[ 1, 3, 0 ]}
{
 }

 }


\chapter{\textcolor{Chapter }{Automata groups}}\label{ch:automgrps}
\logpage{[ 2, 0, 0 ]}
{

\section{\textcolor{Chapter }{Trees}}\logpage{[ 2, 1, 0 ]}
{
 These functions allow to construct and operate with vertices of the trees.

\subsection{\textcolor{Chapter }{VertexNumber}}
\logpage{[ 2, 1, 1 ]}\nobreak
{\noindent\textcolor{FuncColor}{$\Diamond$\ \texttt{VertexNumber( num, lev, deg )\index{VertexNumber@\texttt\
{VertexNumber}}
\label{VertexNumber}
}\hfill{\scriptsize (function)}}\\


 One can naturally enumerate all the vertices of the $lev$-th level of the \mbox{\texttt{deg}}-ary tree by n\
umbers $1,\ldots,\rm{deg}^n$. This function returns the vertex of
this level, which has number \mbox{\texttt{num}}.
\begin{Verbatim}[fontsize=\small,frame=single,label=Example]
  gap> VertexNumber(1,3,2);
  [ 1, 1, 1 ]
  gap> VertexNumber(4,4,3);
  [ 1, 1, 2, 1 ]
\end{Verbatim}
 }



\subsection{\textcolor{Chapter }{NumberOfVertex}}
\logpage{[ 2, 1, 2 ]}\nobreak
{\noindent\textcolor{FuncColor}{$\Diamond$\ \texttt{NumberOfVertex( ver, deg )\index{NumberOfVertex@\texttt{\
NumberOfVertex}}
\label{NumberOfVertex}
}\hfill{\scriptsize (function)}}\\


 Let \mbox{\texttt{ver}} belong to $n$-th level of the \mbox{\texttt{deg}}-ary tree. One can naturally enume\
rate all the vertices of this level by numbers
$1,\ldots,\rm{deg}^n$. This function returns the number, which
corresponds to the vertex \mbox{\texttt{ver}}.
\begin{Verbatim}[fontsize=\small,frame=single,label=Example]
  gap> NumberOfVertex([1,2,1,2],2);
  6
  gap> NumberOfVertex("333",3);
  27
\end{Verbatim}
 }

 }


\section{\textcolor{Chapter }{Tree Automorphisms}}\logpage{[ 2, 2, 0 ]}
{
 These functions allow to construct orbit of vertex.

\subsection{\textcolor{Chapter }{OrbitOfVertex}}
\logpage{[ 2, 2, 1 ]}\nobreak
{\noindent\textcolor{FuncColor}{$\Diamond$\ \texttt{OrbitOfVertex( ver, g[, n] )\index{OrbitOfVertex@\texttt\
{OrbitOfVertex}}
\label{OrbitOfVertex}
}\hfill{\scriptsize (operation)}}\\


 Returns the list of vertics in the orbit of vertex \mbox{\texttt{ver}} under the action of a semigroup gene\
rated by an automorphism \mbox{\texttt{g}}. If \mbox{\texttt{n}} is
specified returns only first \mbox{\texttt{n}} elements of the
orbit. Vertices are defined either as lists with entries from
\texttt{[1..d]}, or as \ strings containing characters
\texttt{1},...,\texttt{d}, where \texttt{d} is the degree of the
tree.
\begin{Verbatim}[fontsize=\small,frame=single,label=Example]
  gap> g:=AutomGroup("t=(1,t)(1,2)");;
  gap> OrbitOfVertex([1,1,1],t);
  [ [ 1, 1, 1 ], [ 2, 1, 1 ], [ 1, 2, 1 ], [ 2, 2, 1 ], [ 1, 1, 2 ], [ 2, 1, 2 ], [ 1, 2, 2 ], [ 2, 2, 2 ] ]
  gap> OrbitOfVertex("111111111111",t,6);
  [ [ 1, 1, 1, 1, 1, 1, 1, 1, 1, 1, 1, 1 ], [ 2, 1, 1, 1, 1, 1, 1, 1, 1, 1, 1, 1 ],
  [ 1, 2, 1, 1, 1, 1, 1, 1, 1, 1, 1, 1 ], [ 2, 2, 1, 1, 1, 1, 1, 1, 1, 1, 1, 1 ],
  [ 1, 1, 2, 1, 1, 1, 1, 1, 1, 1, 1, 1 ], [ 2, 1, 2, 1, 1, 1, 1, 1, 1, 1, 1, 1 ] ]
\end{Verbatim}
 }



\subsection{\textcolor{Chapter }{PrintOrbitOfVertex}}
\logpage{[ 2, 2, 2 ]}\nobreak
{\noindent\textcolor{FuncColor}{$\Diamond$\ \texttt{PrintOrbitOfVertex( ver, g[, n] )\index{PrintOrbitOfVert\
ex@\texttt{PrintOrbitOfVertex}}
\label{PrintOrbitOfVertex}
}\hfill{\scriptsize (operation)}}\\


 Prints the orbit of vertex \mbox{\texttt{ver}} under the action of a semigroup generated by an automorphism\
 \mbox{\texttt{g}}. Each vertex is printed as a string containing characters \texttt{1},...,\texttt{d}, wher\
e \texttt{d} is the degree of the tree. In case of binary tree the symbols ' ' and '\texttt{x}' are used to \
represent \texttt{1} and \texttt{2}. If \mbox{\texttt{n}} is specified only first \mbox{\texttt{n}} elements\
 of the orbit are printed. Vertices are defined either as lists with
entries from \texttt{[1..d]}, or as strings
\begin{Verbatim}[fontsize=\small,frame=single,label=Example]
  gap> g:=AutomGroup("a=(b,a)(1,2),b=(b,a)");;
  gap> PrintOrbitOfVertex("2222222222222222222222222222222",a*b^-2,6);
  xxxxxxxxxxxxxxxxxxxxxxxxxxxxxxx

  x x x x x x x x x x x x x x x x
   xx  xx  xx  xx  xx  xx  xx  xx
  xxx xxx xxx xxx xxx xxx xxx xxx
     xxxx    xxxx    xxxx    xxxx
  gap> h:=AutomGroup("a=(b,1,1)(1,2,3),b=(a,b,a)(1,2)");;
  gap> PrintOrbitOfVertex([1,2,1],b^2);
  121
  132
  123
  131
  122
  133
\end{Verbatim}
 }

 }

 }


\chapter{\textcolor{Chapter }{Contracting groups}}\label{ch:contr}
\logpage{[ 3, 0, 0 ]}
{
 }

 \def\indexname{Index\logpage{[ "Ind", 0, 0 ]}}


\printindex

\newpage
\immediate\write\pagenrlog{["End"], \arabic{page}];}
\immediate\closeout\pagenrlog
\end{document}
