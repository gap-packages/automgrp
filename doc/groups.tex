% This file was created automatically from groups.msk.
% DO NOT EDIT!
%%%%%%%%%%%%%%%%%%%%%%%%%%%%%%%%%%%%%%%%%%%%%%%%%%%%%%%%%%%%%%%%%%
\Chapter{Properties and operations with groups and semigroups}


%%%%%%%%%%%%%%%%%%%%%%%%%%%%%%%%%%%%%%%%%%%%%%%%%%%%%%%%%%%%%%%%%%
\Section{Creation of groups and semigroups}

\>AutomatonGroup( <string>[, <bind_vars>] ) O
\>AutomatonGroup( <list>[, <names>, <bind_vars>] ) O
\>AutomatonGroup( <automaton>[, <bind_vars>] ) O

Creates the self-similar group generated by finite automaton, described by <string>
or <list>, or given as an argument <automaton>.

The <string> is a conventional notation of the form
`name1 = (name11, name12, ..., name1d)perm1, name2 = ...'
where each `name\*' is a name of state or `1', and each `perm\*' is a
permutation written in {\GAP} notation. Trivial permutations may be
omitted. This function ignores whitespace, and states may be separated
by commas or semicolons.

The <list> is a list consisting of $n$ entries corresponding to $n$ states of automaton.
Each entry is of the form $[a_1,\.\.\.,a_d,p]$,
where $d \geq 2$ is the size of the alphabet the group acts on, $a_i$ are `IsInt' in
$\{1,\ldots,n\}$ and
represent the sections of corresponding state at all vertices of the first level of the tree;
and all $p$ are in `SymmetricGroup(<d>)' describes the action of the corresponding state on the
alphabet.

Optional <names> must be a list of names of generators of the group, corresponding to the
states of automaton.
These names are used to display elements of resulted group.

If optional <bind_vars> is `false' the names of generators of the group are not assigned
to the global variables. The default value is `true'. One can use
`AssignGeneratorVariables' function to assign these names later, if they were not assigned
when the group was created.

\beginexample
gap> AutomatonGroup("a = (a, b), b = (a, b)(1,2)");
< a, b >
gap> AutomatonGroup("a=(b, a, 1)(2,3), b=(1, a, b)(1,2,3)");
< a, b >
gap> A:=MealyAutomaton("a=(b,1)(1,2),b=(a,1)");
<automaton>
gap> G:=AutomatonGroup(A);
< a, b >
\endexample

In the second form of this operation the definition of the first group
looks like
\beginexample
gap> AutomatonGroup([ [ 1, 2, ()], [ 1, 2, (1,2) ] ], [ "a", "b" ]);
< a, b >
\endexample
The <bind_vars> argument works as follows
\beginexample
gap> AutomatonGroup("t = (1, t)(1,2)", false);;
gap> t;
Variable: 't' must have a value
gap> AutomatonGroup("t = (1, t)(1,2)", true);;
gap> t;
t
\endexample


\>AutomatonSemigroup( <string>[, <bind_vars>] ) O
\>AutomatonSemigroup( <list>[, <names>, <bind_vars>] ) O
\>AutomatonSemigroup( <automaton>[, <bind_vars>] ) O

Creates the semigroup generated by finite automaton, described by <string>
or <list>, or given as an argument <automaton>.

The <string> is a conventional notation of the form
`name1 = (name11, name12, ..., name1d)trans1, name2 = ...'
where each `name\*' is a name of state or `1', and each `trans\*' is either a
permutation written in {\GAP} notation, or the list defining a transformation
of the alphabet via `Transformation(trans\*)'. Trivial permutations may be
omitted. This function ignores whitespace, and states may be separated
by commas or semicolons.

The <list> is a list consisting of $n$ entries corresponding to $n$ states of automaton.
Each entry is of the form $[a_1,...,a_d,p]$,
where $d \geq 2$ is the size of the alphabet the group acts on, $a_i$ are `IsInt' in
$\{1,\ldots,n\}$ and
represent the sections of corresponding state at all vertices of the first level of the tree;
and each $p$ is a transformation of the alphabet describing the action of the corresponding
state on the alphabet.

Optional <names> and <bind_vars> have the same meaning as in `AutomatonGroup' (see "AutomatonGroup").

\beginexample
gap> AutomatonSemigroup("a = (a, b)[2,2], b = (a, b)(1,2)");
< a, b >
gap> AutomatonSemigroup("a=(b, a, 1)[1,1,3], b=(1, a, b)(1,2,3)");
< a, b >
gap> A:=MealyAutomaton("f0=(f0,f0)(1,2),f1=(f1,f0)[2,2]");
<automaton>
gap> G:=AutomatonSemigroup(A);
< f0, f1 >
\endexample
In the second form of this operation the definition of the second semigroup
looks like
\beginexample
gap> AutomatonSemigroup([ [1,2,Transformation([2,2])], [ 1,2,(1,2)] ], ["a","b"]);
< a, b >
\endexample



\>SelfSimilarGroup( <string>[, <bind_vars>] ) O
\>SelfSimilarGroup( <list>[, <names>, <bind_vars>] ) O
\>SelfSimilarGroup( <automaton>[, <bind_vars>] ) O

Creates the self-similar group generated by the wreath recursion, described by <string>
or <list>, or given as an argument <automaton>.

The <string> is a conventional notation of the form
`name1 = (word11, word12, ..., word1d)perm1, name2 = ...'
where each `name\*' is a name of state, `word\*' is an associative word
over the alphabet consisting of all `name\*', and each `perm\*' is a
permutation written in {\GAP} notation. Trivial permutations may be
omitted. This function ignores whitespace, and states may be separated
by commas or semicolons.

The <list> is a list consisting of $n$ entries corresponding to $n$ generators of a group.
Each entry is of the form $[a_1,\.\.\.,a_d,p]$,
where $d \geq 2$ is the size of the alphabet the group acts on, $a_i$ are the lists
acceptable by `AssocWordByLetterRep' (e.g. if the names of generators are `x', `y' and `z',
then `[1, 1, -2, -2, 1, 3]' will produce `x^2*y^-2*x*z')
representing the sections of corresponding generator at all vertices of the first level of the tree;
and $p$ is in `SymmetricGroup(<d>)' describes the action of the corresponding generator on the
alphabet.

Optional <names> must be a list of names of generators of the group.
These names are used to display the elements of resulted group.

If optional <bind_vars> is `false' the names of generators of the group are not assigned
to the global variables. The default value is `true'. One can use
`AssignGeneratorVariables' function to assign these names later, if they were not assigned
when the group was created.

\beginexample
gap> SelfSimilarGroup("a = (a*b, b^-1), b = (1, b^2*a)(1,2)");
< a, b >
gap> SelfSimilarGroup("a=(b, a, a^-1)(2,3), b=(1, a*b, b)(1,2,3)");
< a, b >
gap> A:=MealyAutomaton("f0=(f0,f0)(1,2),f1=(f1,f0)");
<automaton>
gap> SelfSimilarGroup(A);
< f0, f1 >
\endexample
In the second form of this operation the definition of the first group
looks like
\beginexample
gap> SelfSimilarGroup([ [ [1,2], [-2], ()], [ [], [2,2,1], (1,2) ] ], ["a","b"]);
< a, b >
\endexample
The <bind_vars> argument works as follows
\beginexample
gap> SelfSimilarGroup("t = (t^2, t)(1,2)", false);;
gap> t;
Variable: 't' must have a value
gap> SelfSimilarGroup("t = (t^2, t)(1,2)", true);;
gap> t;
t
\endexample



\>`IsTreeAutomorphismGroup'{IsTreeAutomorphismGroup}@{`IsTreeAutomorphismGroup'} C

The category of groups of tree automorphisms.


\>`IsAutomGroup'{IsAutomGroup}@{`IsAutomGroup'} C

The category of groups generated by finite invertible initial automata
(elements from category `IsAutom').


\>IsAutomatonGroup( <G> ) P

is `true' if generators of <G> coincide with generators
of `GroupOfAutomFamily(UnderlyingAutomFamily(<G>))', which
means that the <G> is generated by the states of underlying automaton
and generators of <G> correspond to the states of this automaton.



In the examples below we will consider the elements of Basilica group and 
Grigorchuk group. First we define them as
\beginexample
gap> Basilica:=AutomatonGroup("u=(v,1)(1,2),v=(u,1)");
< u, v >
gap> GrigorchukGroup:=AutomatonGroup("a=(1,1)(1,2),b=(a,c),c=(a,d),d=(1,b)");
< a, b, c, d >
\endexample

Note also that these groups are predefined in a global variable `AG_Groups' (see "Some predefined groups").

%%%%%%%%%%%%%%%%%%%%%%%%%%%%%%%%%%%%%%%%%%%%%%%%%%%%%%%%%%%%%%%%%%
\Section{Basic properties of groups and semigroups}

\>DegreeOfTree( <obj> ) A

This is a synonym for TopDegreeOfTree~("TopDegreeOfTree") for the case of
regular tree. It is an error to call this method for an object which acts
on a non-regular tree.


\>IsFractal( <G> ) P

Returns whether the group <G> is fractal.
\beginexample
gap> IsFractal(GrigorchukGroup);
true
\endexample


\>IsFractalByWords( <G> ) P

Computes the generators of stabilizers of vertices of the first level
and their projections on these vertices. Returns `true' if  the preimages of these
projections in the free group under canonical epimorphism generate the whole free
group for each stabilizer, and the <G> acts transitively on the first level.
This is sufficient but not necessary condition for <G> to be fractal. See also
`IsFractal' ("IsFractal").

\>`IsSphericallyTransitive( <G> )'{IsSphericallyTransitive![treeautgrp]}@{`IsSphericallyTransitive'!`[treeautgrp]'} P

Returns whether the group <G> is spherically transitive (see~"Short math background").
\beginexample
gap> IsSphericallyTransitive(GrigorchukGroup);
true
\endexample


\>`IsTransitiveOnLevel( <G>, <lev> )'{IsTransitiveOnLevel![treeautgrp]}@{`IsTransitiveOnLevel'!`[treeautgrp]'} O

Returns whether the group <G> acts transitively on level <lev>.
\beginexample
gap> IsTransitiveOnLevel(Group([a,b]),3);
true
gap> IsTransitiveOnLevel(Group([a,b]),4);
false
\endexample


\>IsAutomatonGroup( <G> ) P

is `true' if generators of <G> coincide with generators
of `GroupOfAutomFamily(UnderlyingAutomFamily(<G>))', which
means that the <G> is generated by the states of underlying automaton
and generators of <G> correspond to the states of this automaton.

\>IsSelfSimilar( <G> ) P

Returns whether the group or semigroup <G> is self-similar (see "Short math background").


\>IsContracting( <G> ) A

Given a self-similar group <G> tries to compute whether it is contracting or not.
Only the partial method is implemented (since there is no general algorithm so far).
First it tries to find the nucleus up to size 50 using `FindNucleus'(<G>,50) (see~"FindNucleus"), then
it tries to find the evidence that the group is noncontracting using
`IsNoncontracting'(<G>,10,10) (see~"IsNoncontracting"). If the answer was not found one can try to use
`FindNucleus' and `IsNoncontracting' with bigger tolerances.

\beginexample
gap> IsContracting(Basilica);
true
gap> IsContracting(AutomatonGroup("a=(c,a)(1,2),b=(c,b),c=(b,a)"));
#I  (b*c^-1)^1 has b*a^-1 as a section at vertex [ 2 ]
#I  (b*a^-1)^2 has congutate of a^-1*b as a section at vertex [ 1 ]
false
\endexample


\>IsNoncontracting( <G>[, <max_len>, <depth>] ) F

Tries to show that the group <G> is not contracting.
Enumerates the elements of the group <G> up to length <max_len>
until it finds an element which has a section <g> of infinite order, such that
`OrderUsingSections'(<g>, <depth>) (see "OrderUsingSections")
is infinity and such that <g> stabilizes some vertex and has itself as a
section at this vertex. See also `IsContracting'~("IsContracting").

\beginexample
gap> G:=AutomatonGroup("a=(b,a)(1,2),b=(c,b)(),c=(c,a)");
< a, b, c >
gap> IsNoncontracting(G,10,10);
true
\endexample


\>IsGeneratedByAutomatonOfPolynomialGrowth( <G> ) P

For a group <G> generated by all states of finite automaton (see "IsAutomatonGroup")
determines whether this automaton has polynomial growth in terms of Sidki~\cite{sidki:circuit}.

See also `IsGeneratedByBoundedAutomaton' ("IsGeneratedByBoundedAutomaton" and
`PolynomialDegreeOfGrowthOfAutomaton' ("PolynomialDegreeOfGrowthOfAutomaton").
\beginexample
gap> IsGeneratedByAutomatonOfPolynomialGrowth(Basilica);
true
gap> D:=AutomatonGroup("a=(a,b)(1,2),b=(b,a)");
< a, b >
gap> IsGeneratedByAutomatonOfPolynomialGrowth(D);
false
\endexample


\>IsGeneratedByBoundedAutomaton( <G> ) P

For a group <G> generated by all states of finite automaton (see "IsAutomatonGroup")
determines whether this automaton is bounded in terms of Sidki~\cite{sidki:circuit}.

See also `IsGeneratedByAutomatonOfPolynomialGrowth' ("IsGeneratedByAutomatonOfPolynomialGrowth")
and `PolynomialDegreeOfGrowthOfAutomaton' ("PolynomialDegreeOfGrowthOfAutomaton").
\beginexample
gap> IsGeneratedByBoundedAutomaton(Basilica);
true
gap> C:=AutomatonGroup("a=(a,b)(1,2),b=(b,c),c=(c,1)(1,2)");
< a, b >
gap> IsGeneratedByBoundedAutomaton(C);
false
\endexample


\>PolynomialDegreeOfGrowthOfUnderlyingAutomaton( <G> ) A

For a group <G> generated by all states of finite automaton (see "IsAutomatonGroup")
of polynomial growth in terms of Sidki~\cite{sidki:circuit} determines the degree of
polynomial growth of this automaton. This degree is 0 if and only if the automaton is bounded.
If the growth of automaton is exponential returns `fail'.

See also `IsGeneratedByAutomatonOfPolynomialGrowth' ("IsGeneratedByAutomatonOfPolynomialGrowth")
and `IsGeneratedByBoundedAutomaton' ("IsGeneratedByBoundedAutomaton").
\beginexample
gap> PolynomialDegreeOfGrowthOfUnderlyingAutomaton(Basilica);
0
gap> C:=AutomatonGroup("a=(a,b)(1,2),b=(b,c),c=(c,1)(1,2)");
< a, b >
gap> PolynomialDegreeOfGrowthOfUnderlyingAutomaton(C);
2
\endexample


\>UnderlyingAutomaton( <G> ) A

For a group (or semigroup) <G> returns an automaton generating a
self-similar group (or semigroup) containing <G>.
\beginexample
gap> GS:=AutomatonSemigroup("x=(x,y)[1,1],y=(y,y)(1,2)");
< x, y >
gap> UnderlyingAutomaton(GS);
<automaton>
gap> A:=UnderlyingAutomaton(GS);
<automaton>
gap> Print(A);
a1 = (a1, a2)[ 1, 1 ], a2 = (a2, a2)[ 2, 1 ]
\endexample
For a subgroup of Basilica group we get the automaton generating Basilica group.
\beginexample
gap> H:=Group([u*v^-1,v^2]);
< u*v^-1, v^2 >
gap> Print(UnderlyingAutomaton(H));
a1 = (a1, a1), a2 = (a3, a1)(1,2), a3 = (a2, a1)
\endexample




%%%%%%%%%%%%%%%%%%%%%%%%%%%%%%%%%%%%%%%%%%%%%%%%%%%%%%%%%%%%%%%%%%
\Section{Operations with groups and semigroups}

\>PermGroupOnLevel( <G>, <k> ) O

Returns group of permutations induced by action of group <G> at the <k>-th
level.
\beginexample
gap> PermGroupOnLevel(Basilica,4);
Group([ (1,11,3,9)(2,12,4,10)(5,13)(6,14)(7,15)(8,16), (1,6,2,5)(3,7)(4,8) ])
gap> H:=PermGroupOnLevel(Group([u,v^2]),4);
Group([ (1,11,3,9)(2,12,4,10)(5,13)(6,14)(7,15)(8,16), (1,2)(5,6) ])
gap> Size(H);
64
\endexample


\>StabilizerOfLevel( <G>, <k> ) O

Returns the stabilizer of the <k>-th level.
\beginexample
gap> StabilizerOfLevel(Basilica, 2);
< u*v^2*u^-1, u*v*u*v^2*u^-1*v^-1*u^-1, v^2, v*u^2*v^-1, u*v*u^2*v^-1*u^-1, u^
2, v*u*v*u*v^-1*u^-1*v^-1*u^-1 >
\endexample


\>StabilizerOfFirstLevel( <G> ) A

Returns the stabilizer of the first level, see also~"StabilizerOfLevel".
\beginexample
gap> StabilizerOfFirstLevel(Basilica);
< u^2, u*v*u^-1, v >
\endexample


\>StabilizerOfVertex( <G>, <v> ) O

Returns stabilizer of the vertex <v>. Here <v> can be a list represnting a
vertex, or a positive intger representing a vertex at the first level.
\beginexample
gap> StabilizerOfVertex(Basilica,[1,2,1]);
< v*u^4*v^-1, v*u^2*v^2*u^-2*v^-1, v^2, u^2, v*u^2*v*u^2*v^-1*u^-2*v^-1, u*v*u^
-1, v*u^-1*v*u*v^-1, v*u^2*v*u*v*u^-1*v^-1*u^-2*v^-1 >
\endexample


\>FixesLevel( <obj>, <lev> ) O

Returns whether <obj> fixes level <lev>, i.e. fixes every vertex at the level
<lev>.


\>FixesVertex( <obj>, <v> ) O

Returns whether <obj> fixes vertex <v>. Vertex <v> may be given as a list, or as
a positive integer, in which case it denotes <v>-th vertex at the first
level.


\>Projection( <G>, <v> ) O
\>ProjectionNC( <G>, <v> ) O

Returns projection of the group <G> at the vertex <v>. The group <G> must fix the
the vertex <v>, otherwise `Error'() will be called. The operation `ProjectionNC' does the
same thing, except it does not check whether <G> fixes vertex <v>.
\beginexample
gap> Projection(StabilizerOfVertex(Basilica,[1,2,1]),[1,2,1]);
< v, u >
\endexample


\>ProjStab( <G>, <v> ) O

Returns projection of the stabilizer of <v> at itself. It is a shortcut for
`Projection'(`StabilizerOfVertex'(G, v), v) (see "Projection",
"StabilizerOfVertex").
\beginexample
gap> ProjStab(Basilica,[1,2,1]);
< v, u >
\endexample


\>FindGroupRelations( <G>[, <max_len>, <max_num_rels>] ) O
\>FindGroupRelations( <subs_words>[, <names>, <max_len>, <max_num_rels>] ) O

Finds group relations between the generators of group <G>
or in the group generated by <subs_words>. Stops after investigating all words
of length up to <max_len> elements or when it finds <max_num_rels>
relations. The optional argument <names> is a list of names of generators of the same length
as <subs_words>. If this argument is given the relations are given in terms of these names.
Otherwise they are given in terms of the elements of group generated by <subs_words>.
If <max_len> or <max_num_rels> are not specified, they are assumed to be `infinity'.
This operation can be applied for any group, not only for group generated by automata.
\beginexample
gap> FindGroupRelations(Basilica,5);
#I  v*u*v*u^-1*v^-1*u*v^-1*u^-1
#I  v*u*v^2*u^-1*v^-1*u*v^-2*u^-1
#I  v^2*u*v*u^-1*v^-2*u*v^-1*u^-1
[ v*u*v*u^-1*v^-1*u*v^-1*u^-1, v*u*v^2*u^-1*v^-1*u*v^-2*u^-1,
  v^2*u*v*u^-1*v^-2*u*v^-1*u^-1 ]
gap> FindGroupRelations([u*v^-1,v*u],["x","y"],5);
#I  y*x^2*y*x^-1*y^-2*x^-1
[ y*x^2*y*x^-1*y^-2*x^-1 ]
gap> FindGroupRelations([u*v^-1,v*u],5);
#I  v*u^2*v^-1*u^2*v*u^-2*v^-1*u^-2
[ v*u^2*v^-1*u^2*v*u^-2*v^-1*u^-2 ]
gap> FindGroupRelations([(1,2)(3,4),(1,2,3)],["x","y"]);
#I  x^2
#I  y^-3
#I  y^-1*x*y^-1*x*y^-1*x
#I  y*x*y^-1*x*y^-1*x*y
[ x^2, y^-3, y^-1*x*y^-1*x*y^-1*x, y*x*y^-1*x*y^-1*x*y ]
\endexample


\>FindSemigroupRelations( <G>[, <max_len>, <max_num_rels>] ) O
\>FindSemigroupRelations( <subs_words>[, <names>, <max_len>, <max_num_rels>] ) O

Finds semigroup relatoins between the generators of the group or semigroup <G>,
or in the semigroup generated by <subs_words>. Arguments have the same meaning
as in `FindGroupRelations'~("FindGroupRelations"). It returns the list of pairs of equal words.
In order to make the list of relations shorter
it also tries to remove the relations that can
be derived from the known ones. Note, that by default the trivial automorphism is
not inclded in every semigroup. So if one needs to find the relations of the form
$w=1$ be sure to define <G> as a monoid or to include the trivial automorphism
into <subs_words> (for instance, as `One(g)' for any element `g' acting on the same
tree).
This operation can be applied for any semigroup, not only for semigroup generated by automata.
\beginexample
gap> FindSemigroupRelations([u*v^-1,v*u],["x","y"],6);
#I  y*x^2*y=x*y^2*x
#I  y*x^3*y^2=x^2*y^3*x
#I  y^2*x^3*y=x*y^3*x^2
[ [ y*x^2*y, x*y^2*x ], [ y*x^3*y^2, x^2*y^3*x ], [ y^2*x^3*y, x*y^3*x^2 ] ]
gap> FindSemigroupRelations([u*v^-1,v*u],6);
#I  v*u^2*v^-1*u^2=u^2*v*u^2*v^-1
#I  v*u^2*v^-1*u*v^-1*u^2*v*u=u*v^-1*u^2*v*u*v*u^2*v^-1
#I  v*u*v*u^2*v^-1*u*v^-1*u^2=u^2*v*u*v*u^2*v^-1*u*v^-1
[ [ v*u^2*v^-1*u^2, u^2*v*u^2*v^-1 ],
  [ v*u^2*v^-1*u*v^-1*u^2*v*u, u*v^-1*u^2*v*u*v*u^2*v^-1 ],
  [ v*u*v*u^2*v^-1*u*v^-1*u^2, u^2*v*u*v*u^2*v^-1*u*v^-1 ] ]
gap> x:=Transformation([1,1,2]);;
gap> y:=Transformation([2,2,3]);;
gap> FindSemigroupRelations([x,y],["x","y"]);
#I  y*x=x
#I  y^2=y
#I  x^3=x^2
#I  x^2*y=x*y
[ [ y*x, x ], [ y^2, y ], [ x^3, x^2 ], [ x^2*y, x*y ] ]
\endexample


\>FindElement( <G>, <func>, <val>, <max_len> ) O
\>FindElements( <G>, <func>, <val>, <max_len> ) O

The first function enumerates elements of the group (semigroup, monoid) <G> until it finds
an element $g$ of length at most <max_len>, for which <func>($g$)=<val>. Returns $g$ if
such an element was found and `fail' otherwise.

The second function enumerates elements of the group (semigroup, monoid) of length at most <max_len>
and returns the list of elements $g$, for which <func>($g$)=<val>.

These functions are based on `Iterator' operation (see "Iterator"), so can be applied in
more general settings whenever \GAP knows how to solve word problem in the group.
The following examlpe illustrates how to find an element of order 16 in
Grigorchuk group and the list of all such elements of length at most 5.
\beginexample
gap> FindElement(GrigorchukGroup,Order,16,5);
a*b
gap> FindElements(GrigorchukGroup,Order,16,5);
[ a*b, b*a, c*a*d, d*a*c, a*b*a*d, a*c*a*d, a*d*a*b, a*d*a*c, b*a*d*a, c*a*d*a,
  d*a*b*a, d*a*c*a, a*c*a*d*a, a*d*a*c*a, b*a*b*a*c, b*a*c*a*c, c*a*b*a*b,
  c*a*c*a*b ]
\endexample


\>FindElementOfInfiniteOrder( <G>, <max_len>, <depth> ) O
\>FindElementsOfInfiniteOrder( <G>, <max_len>, <depth> ) O

The first function enumerates elements of the group <G> up to length <max_len>
until it finds an element $g$ of infinite order, such that
`OrderUsingSections'($g$,<depth>) (see "OrderUsingSections") is `infinity'.
In other words all sections of every element up to depth <depth> are
investigated. In case if the element belongs to the group generated by bounded
automaton (see "IsGeneratedByBoundedAutomaton") one can set <depth> to be `infinity'.

The second function returns the list of all such elements up to length <max_len>.

\beginexample
gap> G:=AutomatonGroup("a=(1,1)(1,2),b=(a,c),c=(b,1)");
< a, b, c >
gap> FindElementOfInfiniteOrder(G,5,10);
a*b*c
\endexample


\>Growth( <G>, <max_len> ) O

Returns the list of the first values of the growth function of a group
(semigroup, monoid) <G>.
If <G> is a monoid it computes the growth function at $\{0,1,\ldots,<max_len>\}$,
and for a semigroup without identity at $\{0,1,\ldots,<max_len>\}$.
\beginexample
gap> Growth(GrigorchukGroup,7);
#I  Length not greater than 2: 11
#I  Length not greater than 3: 23
#I  Length not greater than 4: 40
#I  Length not greater than 5: 68
#I  Length not greater than 6: 108
#I  Length not greater than 7: 176
[ 1, 5, 11, 23, 40, 68, 108, 176 ]
gap> H:=AutomatonSemigroup("a=(a,b)[1,1],b=(b,a)(1,2)");
< a, b >
gap> Growth(H,6);
[ 2, 6, 14, 30, 62, 126 ]
\endexample


\>ListOfElements( <G>, <max_len> ) O

Returns the list of all different elements of a group (semigroup, monoid) 
<G> up to length <max_len>.
\beginexample
gap> ListOfElements(GrigorchukGroup,3);
[ 1, a, b, c, d, a*b, a*c, a*d, b*a, c*a, d*a, a*b*a, a*c*a, a*d*a, b*a*b, b*a*c,
  b*a*d, c*a*b, c*a*c, c*a*d, d*a*b, d*a*c, d*a*d ]
\endexample


\>FindNucleus( <G>[, <max_nucl>] ) O

Given a self-similar group <G> it tries to find its nucleus. If the group
is not contracting it will loop forever. When it finds the nucleus it returns
the triple [`GeneratingSetWithNucleus'(<G>), `GroupNucleus'(<G>),
`GeneratingSetWithNucleusAutom'(<G>)] (see "GeneratingSetWithNucleus",
"GroupNucleus", "GeneratingSetWithNucleusAutom").

If <max_nucl> is given stops after finding <max_nucl> elements that need to be in
the nucleus and returns `fail' if the nucleus was not found.

Use `IsNoncontracting'~(see "IsNoncontracting") to try to show that <G> is
noncontracting.

\beginexample
gap> FindNucleus(Basilica);
[ [ 1, u, v, u^-1, v^-1, u^-1*v, v^-1*u ], [ 1, u, v, u^-1, v^-1, u^-1*v, v^-1*u ]
    , <automaton> ]
\endexample


\>LevelOfFaithfulAction( <G> ) A
\>LevelOfFaithfulAction( <G>, <max_lev> ) A

For a given finite self-similar group <G> determines the smallest level of
the tree, where <G> acts faithfully, i.e. the stabilizer of this level in <G>
is trivial. The idea here is that for self-similar group all nontrivial level
stabilizers are different. If <max_lev> is given it finds only first <max_lev>
quotients by stabilizers and if all of them have different size returns `fail'.
If <G> is infinite and <max_lev> is not specified will loop forever.

See also `IsomorphismPermGroup' ("IsomorphismPermGroup").
\beginexample
gap> H:=AutomatonGroup("a=(a,a)(1,2),b=(a,a),c=(b,a)(1,2)");
< a, b, c >
gap> LevelOfFaithfulAction(H);
3
gap> LevelOfFaithfulAction(AddingMachine,10);
fail
\endexample


\>IsomorphismPermGroup( <G> ) O
\>IsomorphismPermGroup( <G>, <max_lev> ) O

For a given finite group <G> generated by initial automata (see "IsAutomGroup")
computes an isomorphism from <G> into a finite permutational group.
If <G> is not known to be self-similar (see "IsSelfSimilar") the isomorphism is based on the
regular representation, which works generally much slower. If <G> is self-similar
there is a level of the tree (see "LevelOfFaithfulAction"), where <G> acts faithfully.
The corresponding representation is returned in this case. If <max_lev> is given
it finds only first <max_lev> quotients by stabilizers and if all of them have
different size returns `fail'.
If <G> is infinite and <max_lev> is not specified will loop forever.

For example, consider a subgroup $\langle a,b\rangle$ of Grigorchuk group.
\beginexample
gap> f:=IsomorphismPermGroup(Group(a,b));
[ a, b ] -> [ (1,2)(3,5)(4,6)(7,9)(8,10)(11,13)(12,14)(15,17)(16,18)(19,21)(20,
    22)(23,25)(24,26)(27,29)(28,30)(31,32), (1,3)(2,4)(5,7)(6,8)(9,11)(10,12)(13,
    15)(14,16)(17,19)(18,20)(21,23)(22,24)(25,27)(26,28)(29,31)(30,32) ]
gap> Size(Image(f));
32
gap> H:=AutomatonGroup("a=(a,a)(1,2),b=(a,a),c=(b,a)(1,2)");
< a, b, c >
gap> f1:=IsomorphismPermGroup(H);
[ a, b, c ] -> [ (1,8)(2,7)(3,6)(4,5), (1,4)(2,3)(5,8)(6,7), (1,6,3,8)(2,5,4,7) ]
gap> Size(Image(f1));
16
\endexample


\>MarkovOperator( <G>, <lev> ) F

Computes the matrix of Markov operator related to group <G> on the <lev>-th level
of a tree. If the group <G> is generated by $g_1,g_2,\ldots,g_n$ then the Markov operator
is defined as $(`PermOnLevelAsMatrix'(g_1)+\cdots+`PermOnLevelAsMatrix'(g_d)+
`PermOnLevelAsMatrix'(g_1^{-1})+\cdots+`PermOnLevelAsMatrix'(g_d^{-1}))/(2*d)$. See also
`PermOnLevelAsMatrix' ("PermOnLevelAsMatrix").
\beginexample
gap> G:=AutomatonGroup("a=(a,b)(1,2),b=(a,b)");
< a, b >
gap> MarkovOperator(G,3);
[ [ 0, 0, 1/4, 1/4, 0, 1/4, 0, 1/4 ], [ 0, 0, 1/4, 1/4, 1/4, 0, 1/4, 0 ],
  [ 1/4, 1/4, 0, 0, 1/4, 0, 1/4, 0 ], [ 1/4, 1/4, 0, 0, 0, 1/4, 0, 1/4 ],
  [ 0, 1/4, 1/4, 0, 0, 1/2, 0, 0 ], [ 1/4, 0, 0, 1/4, 1/2, 0, 0, 0 ],
  [ 0, 1/4, 1/4, 0, 0, 0, 1/2, 0 ], [ 1/4, 0, 0, 1/4, 0, 0, 0, 1/2 ] ]
\endexample




\>AbelImage( <obj> ) A

Returns image of <obj> in canonical projection onto abelianization of
the full group of tree automorphisms, represented as a subgroup of additive
group of rational functions.
XXX it doesn't make sense for non-invertible automata, does it?


\>DiagonalAction( <fam>[, <k>] ) O

For a given automaton group <G> acting on alphabet $X$ and corresponding family 
<fam> of automata one can consider the action of $<G>^<k>$ on $X^<k>$ defined by
$(x_1,x_2,\ldots, x_k)^{(g_1,g_2,\ldots,g_k)}=(x_1^{g_1},x_2^{g_2},\ldots,x_k^{g_k})$.
This function constructs a self-similar group, which encodes this action. If
<k> is not given it is assumed to be $2$.
\beginexample
gap> S:=DiagonalAction(UnderlyingAutomFamily(Basilica));
< uu, uv, u1, vu, vv, v1, 1u, 1v >
gap> Decompose(uu);
(vv, v1, 1v, 1)(1,4)(2,3)
\endexample


\>MultAutomAlphabet( <fam> ) O



\>UnderlyingAutomFamily( <G> ) A

Returns the family to which elements of <G> belong.


\>UnderlyingFreeSubgroup( <G> ) AM


\>UnderlyingFreeGenerators( <G> ) AM
\>UnderlyingFreeMonoid( <G> ) A
\>UnderlyingFreeGroup( <G> ) A


\>IndexInFreeGroup( <G> ) AM


\>MihailovaSystem( <G> ) AM




%%%%%%%%%%%%%%%%%%%%%%%%%%%%%%%%%%%%%%%%%%%%%%%%%%%%%%%%%%%%%%%%%%
\Section{Self-similar groups and semigroups defined by wreath recursion}

\>IsomorphicAutomGroup( <G> ) AM

In case <G> is finite-state returns a group generated by automata, isomorphic to <G>,
Which is a subgroup of `UnderlyingAutomatonGroup'(<G>) (see "UnderlyingAutomatonGroup").
The natural isomorphism between <G> and `IsomorphicAutomGroup'(<G>) is stored in the
attribute `MonomorphismToAutomatonGroup'(<G>) ("MonomorphismToAutomatonGroup").
\beginexample
gap> R := SelfSimilarGroup("a=(a^-1*b,b^-1*a)(1,2), b=(a^-1,b^-1)");
< a, b >
gap> UR := UnderlyingAutomatonGroup(R);
< a1, a2, a4, a5 >
gap> IR := IsomorphicAutomGroup(R);
< a1, a5 >
gap> hom := MonomorphismToAutomatonGroup(R);
MappingByFunction( < a, b >, < a1, a5 >, function( a ) ... end, function( b ) ... end )
gap> (a*b)^hom;
a1*a5
gap> PreImagesRepresentative(hom,last);
a*b
gap> List( GeneratorsOfGroup (UR),x -> PreImagesRepresentative( hom, x));
[ a, a^-1*b, b^-1*a, b ]
\endexample

All these operations work also for the subgroups of groups generated by `SelfSimilarGroup'.
("SelfSimilarGroup").
\beginexample
gap> T:=Group([b*a,a*b]);
< b*a, a*b >
gap> IT := IsomorphicAutomGroup(T);
< a1, a4 >
\endexample
Note, that different groups have different `UnderlyingAutomGroup' attributes. For example,
the generator `a1' of group `IT' above is different from the generator `a1' of group `IR'.



\>UnderlyingAutomatonGroup( <G> ) AM

In case <G> is finite-state returns a self-similar closure of <G> as a group
generated by automaton.
The natural monomorphism from <G> and `UnderlyingAutomatonGroup'(<G>) is stored in the
attribute `MonomorphismToAutomatonGroup'(<G>) ("MonomorphismToAutomatonGroup"). If
<G> is created by `SelfSimilarGroup' (see "SelfSimilarGroup"), then the self-similar closure
of <G> coincides with <G>, so one can use `MonomorphismToAutomatonGroup'(<G>) to
get preimages of elements of `UnderlyingAutomatonGroup'(<G>) in <G>. See the example for
`IsomorphicAutomGroup' ("IsomorphicAutomGroup").



\>MonomorphismToAutomatonGroup( <G> ) AM

In case <G> is finite-state returns a monomorphism from <G> into `UnderlyingAutomatonGroup'(<G>)
(see "UnderlyingAutomatonGroup"). If <G> is created by `SelfSimilarGroup' (see "SelfSimilarGroup"),
then one can use `MonomorphismToAutomatonGroup'(<G>) to
get preimages of elements of `UnderlyingAutomatonGroup'(<G>) in <G>. See the example for
`IsomorphicAutomGroup' ("IsomorphicAutomGroup").





%%%%%%%%%%%%%%%%%%%%%%%%%%%%%%%%%%%%%%%%%%%%%%%%%%%%%%%%%%%%%%%%%%
\Section{Contracting groups}

\>GroupNucleus( <G> ) AM

Tries to compute the <nucleus> (the minimal set that need not contain original
generators) of a self-similar group <G>. It uses `FindNucleus' (see "FindNucleus")
operation and behaves accordingly: if the group is not contracting it will loop
forever. See also `GeneratingSetWithNucleus' ("GeneratingSetWithNucleus").

\beginexample
gap> GroupNucleus(Basilica);
[ e, u, v, u^-1, v^-1, u^-1*v, v^-1*u ]
\endexample


\>GeneratingSetWithNucleus( <G> ) AM

Tries to compute the generating set of the group which includes original
generators and the <nucleus> (the minimal set that need not contain original
generators) of a self-similar group <G>. It uses `FindNucleus' operation
and behaves accordingly: if the group is not contracting
it will loop forever (modulo memory constraints, of course).
See also `GroupNucleus' ("GroupNucleus").

\beginexample
gap> GeneratingSetWithNucleus(Basilica);
[ e, u, v, u^-1, v^-1, u^-1*v, v^-1*u ]
\endexample


\>GeneratingSetWithNucleusAutom( <G> ) AM

Computes automaton of the generating set that includes nucleus of the contracting group <G>.
See also `GeneratingSetWithNucleus' ("GeneratingSetWithNucleus").
\beginexample
gap> B_autom:=GeneratingSetWithNucleusAutom(Basilica);
<automaton>
gap> Print(B_autom);
a1 = (a1, a1), a2 = (a3, a1)(1,2), a3 = (a2, a1), a4 = (a1, a5)
(1,2), a5 = (a4, a1), a6 = (a1, a7)(1,2), a7 = (a6, a1)(1,2)
\endexample


\>ContractingLevel( <G> ) AM

Given a contracting group <G> with nucleus $N$, stored in
`GeneratingSetWithNucleus'(<G>) (see "GeneratingSetWithNucleus") computes the
minimal level $n$, such that for every vertex $v$ of the $n$-th
level and all $g, h \in N$ the section $gh|_v \in N$.

In case if it is not known whether <G> is contracting it first tries to compute
the nucleus. If <G> is happened to be noncontracting, it will loop forever. One can
also use `IsNoncontracting' (see "IsNoncontracting") or `FindNucleus' (see
"FindNucleus") directly.
\beginexample
gap> ContractingLevel(GrigorchukGroup);
1
gap> ContractingLevel(Basilica);
2
\endexample


\>ContractingTable( <G> ) AM

Given a contracting group <G> with nucleus $N$ of size $k$, stored in
`GeneratingSetWithNucleus'(<G>)~(see "GeneratingSetWithNucleus")
computes the $k\times k$ table, whose
[i][j]-th entry contains decomposition of $N$[i]$N$[j] on
the `ContractingLevel'(<G>) level~(see "ContractingLevel"). By construction the sections of
$N$[i]$N$[j] on this level belong to $N$. This table is used in the
algorithm solving the word problem in polynomial time.

In case if it is not known whether <G> is contracting it first tries to compute
the nucleus. If <G> is happened to be noncontracting, it will loop forever. One can
also use `IsNoncontracting' (see "IsNoncontracting") or `FindNucleus' (see
"FindNucleus") directly.
\beginexample
gap> ContractingTable(GrigorchukGroup);
[ [ (1, 1), (1, 1)(1,2), (a, c), (a, d), (1, b) ],
  [ (1, 1)(1,2), (1, 1), (c, a)(1,2), (d, a)(1,2), (b, 1)(1,2) ],
  [ (a, c), (a, c)(1,2), (1, 1), (1, b), (a, d) ],
  [ (a, d), (a, d)(1,2), (1, b), (1, 1), (a, c) ],
  [ (1, b), (1, b)(1,2), (a, d), (a, c), (1, 1) ] ]
\endexample


\>UseContraction( <G> ) O
\>DoNotUseContraction( <G> ) O

For a contracting automaton group <G> these two operations determine whether to 
use the algorithm
of polynomial complexity solving the word problem in the group. By default
it is set to <true> as soon as the nucleus of the group was computed. Sometimes
when the nucleus is very big, the standard algorithm of exponential complexity
is faster for short words, but this heavily depends on the group. Therefore
the decision on which algorithm to use is left to the user. To use the
exponential algorithm one can use the second operation `DoNotUseContraction'(<G>).

Below we provide an example which shows that both methods can be of use.
\beginexample
gap> G:=AutomatonGroup("a=(b,b)(1,2),b=(c,a),c=(a,a)");;
gap> IsContracting(G);
true
gap> Length(GroupNucleus(G));
41
gap> Order(a); Order(b); Order(c);
2
2
2
gap> UseContraction(G);
true
gap> H:=Group(a*b,b*c);;
gap> St2:=StabilizerOfLevel(H,2);time;
< b*c*b*c, b^-1*a^-1*b*c*b^-1*a^-1*c^-1*b^-1, a*b*a*b*a*b*a*b, a*b^2*c*a*b*c^-1*b^
-1, a*b^2*c*b*c*b^-1*a^-1, b*c*a*b^2*c*a*b, b*c*a*b*a*b*c^-1*b^-2*a^-1*b^-1*a^
-1, a*b*a*b^2*c*a*b*c^-1*b^-2*a^-1, a*b*a*b^2*c*b*c*b^-1*a^-1*b^-1*a^-1 >
741
gap> IsAbelian(St2);time;
true
11977
gap> DoNotUseContraction(G);
true
gap> H:=Group(a*b,b*c);
gap> St2:=StabilizerOfLevel(H,2);;time;
240
gap> IsAbelian(St2);time;
true
542060
\endexample
Here we show that the group <G> is virtually abelian. First we check that the group
is contracting. Then we see that the size of the nucleus is 41. Since all of generators have
order 2, the subgroup $H = \langle ab,bc \rangle$ has index 2 in <G>. Now we compute
the stabilizer of the second level in $H$ and verify, that it is abelian by 2 methods:
with and without using the contraction. We see, that the time required to compute the stabilizer
is approximately the same in both methods, while verification of commutativity works much faster
with contraction. Here it was enough to consider the first level stabilizer, but the difference
in performance of two methods is better seen for the second level stabilizer.






%%%%%%%%%%%%%%%%%%%%%%%%%%%%%%%%%%%%%%%%%%%%%%%%%%%%%%%%%%%%%%%%%%
