% This file was created automatically from elements.msk.
% DO NOT EDIT!
%%%%%%%%%%%%%%%%%%%%%%%%%%%%%%%%%%%%%%%%%%%%%%%%%%%%%%%%%%%%%%%%%%
\Chapter{Properties and operations with group and semigroup elements}


%%%%%%%%%%%%%%%%%%%%%%%%%%%%%%%%%%%%%%%%%%%%%%%%%%%%%%%%%%%%%%%%%%
\Section{Creation of tree automorphisms and homomorphisms}

\>TreeAutomorphism( <states>, <perm> ) O

Constructs the tree automorphism with states on the first level given by the
argument <states> and acting
on the first level as the permutation <perm>. The <states> must
belong to the same family.
\beginexample
gap> L := AutomatonGroup("p=(p,q)(1,2), q=(p,q)");
< p, q >
gap> r := TreeAutomorphism([p, q, p, q^2],(1,2)(3,4));
(p, q, p, q^2)(1,2)(3,4)
gap> t := TreeAutomorphism([q, 1, p*q, q],(1,2));
(q, 1, p*q, q)(1,2)
gap> r*t;
(p, q^2, p*q, q^2*p*q)(3,4)
\endexample


\>Representative( <word>, <fam> ) O
\>Representative( <word>, <a> ) O

Given an associative word <word> constructs the tree homomorphism from the family
<fam>, or to which homomorphism <a> belongs. This function is useful when
one needs to make some operations with associative words. See also `Word' ("Word").
\beginexample
gap> L := AutomatonGroup("p=(p,q)(1,2), q=(p,q)");
< p, q >
gap> F := UnderlyingFreeGroup(L);
<free group on the generators [ p, q ]>
gap> r := Representative(F.1*F.2^2, p);
p*q^2
gap> Decompose(r);
(p*q^2, q*p^2)(1,2)
gap> H := SelfSimilarGroup("x=(x*y,x)(1,2), y=(x^-1,y)");
< x, y >
gap> F := UnderlyingFreeGroup(H);
<free group on the generators [ x, y ]>
gap> r := Representative(F.1^-1*F.2, x);
x^-1*y
gap> Decompose(r);
(x^-1*y, y^-1*x^-2)(1,2)
\endexample


%Declaration{AutomFamily}


%%%%%%%%%%%%%%%%%%%%%%%%%%%%%%%%%%%%%%%%%%%%%%%%%%%%%%%%%%%%%%%%%%
\Section{Properties and attributes of tree automorphisms and homomorphisms}

\>`IsSphericallyTransitive( <a> )'{IsSphericallyTransitive![treehom]}@{`IsSphericallyTransitive'!`[treehom]'} P

Returns whether the action of <a> is spherically transitive (see "Short math background").


\>`IsTransitiveOnLevel( <a>, <lev> )'{IsTransitiveOnLevel![treehom]}@{`IsTransitiveOnLevel'!`[treehom]'} O

Returns whether <a> acts transitively on level <lev> of the tree.



\>IsOne( <a> ) O

Returns whether automorphism <a> acts trivially on the tree. For contracting groups see also 
`UseContraction' ("UseContraction") and `IsOneContr' ("IsOneContr").
\beginexample
gap> L := AutomatonGroup("p=(p,q)(1,2), q=(p,q)");
< p, q >
gap> IsOne(q*p^-1*q*p^-1);
true
\endexample

\>IsOneContr( <a> ) F

Returns `true' if <a> is trivial automorphism and `false' otherwise. Works for
contracting groups only. Uses polynomial time algorithm.




\>Order( <a> ) O

Computes the order of the automorphism <a>. In some cases it does not stop. Works always (modulo memory 
restrictions) for groups generated by bounded automata.

If `InfoLevel' of `InfoAutomGrp' is greater than or equal to 3 (one can set it by 
`SetInfoLevel( InfoAutomGrp, 3)') and the element has infinite order, then the proof of this fact 
is printed.

\beginexample
gap> L := AutomatonGroup("p=(p,q)(1,2), q=(p,q)");                             
< p, q >
gap> Basilica := AutomatonGroup( "u=(v,1)(1,2), v=(u,1)" );
< u, v >
gap> Order(p*q^-1);
2
gap> SetInfoLevel( InfoAutomGrp, 3);
gap> Order( u^35*v^-12*u^2*v^-3 );
#I  (u^35*v^-12*u^2*v^-3)^68719476736 has conjugate of u^2*v^-3*u^35*v^
-12 as a section at vertex [ 1, 1, 1, 1, 1, 1, 1, 1, 1, 1, 1, 1, 1, 1, 1, 1, 1,
  1, 1, 1, 1, 1, 1, 1, 1, 1, 1, 1, 1, 1, 1, 1, 1, 1, 1, 1 ]
infinity
\endexample

\>OrderUsingSections( <a>[, <max_depth>] ) O

Tries to compute the order of the element <a> by looking at its sections
of depth up to <max_depth>-th level.
If <max_depth> is omitted it is assumed to be `infinity', but then it may not stop. Also note,
that if <max_depth> is not given, it searches the tree in depth first and may be trapped
in some infinite ray, while specifying finite <max_depth> may produce a result by looking at
a section not in that ray.
For bounded automata it will always produce a result.

If `InfoLevel' of `InfoAutomGrp' is greater than
or equal to 3 (one can set it by `SetInfoLevel( InfoAutomGrp, 3)')
and the element has infinite order, then the proof of this fact is printed.

\beginexample
gap> GrigorchukGroup := AutomatonGroup("a=(1,1)(1,2),b=(a,c),c=(a,d),d=(1,b)");
< a, b, c, d >
gap> OrderUsingSections( a*b*a*c*b );
16
gap> Basilica := AutomatonGroup( "u=(v,1)(1,2), v=(u,1)" );
< u, v >
gap> SetInfoLevel( InfoAutomGrp, 3);
gap> OrderUsingSections( u^23*v^-2*u^3*v^15, 10 );
#I  v^13*u^15 is obtained from (u^23*v^-2*u^3*v^15)^1
    by taking sections and cyclic reductions at vertex [ 1 ]
#I  v^13*u^15 is obtained from (v^13*u^15)^4
    by taking sections and cyclic reductions at vertex [ 1, 1 ]
infinity
gap> OrderUsingSections( u^23*v^-2*u^3*v^15, 2 );
fail
gap> G := AutomatonGroup("a=(c,a)(1,2), b=(b,c), c=(b,a)");
< a, b, c >
gap> OrderUsingSections(b,10);
#I  b*c*a^2*b^2*c*a acts transitively on levels and is obtained from (b)^8
    by taking sections and cyclic reductions at vertex
[ 2, 2, 1, 1, 1, 1, 2, 2, 1, 1 ]
infinity
\endexample


\>Perm( <a>[, <lev>] ) O

Returns the permutation induced by the tree automorphism <a> on the level <lev>
(or first level if <lev> is not given). See also
`TransformationOnLevel'~("TransformationOnLevel").


\>PermOnLevel( <a>, <k> ) O

Does the same thing as `Perm'~("Perm").


\>PermOnLevelAsMatrix( <g>, <lev> ) F

Computes the action of the element <g> of a group on the <lev>-th level as a permutational matrix, in
which the i-th row contains 1 at the position i^<g>.
\beginexample
gap> L := AutomatonGroup("p=(p,q)(1,2), q=(p,q)");
< p, q >
gap> PermOnLevel(p*q,2);
(1,4)(2,3)
gap> PermOnLevelAsMatrix(p*q, 2);
[ [ 0, 0, 0, 1 ], [ 0, 0, 1, 0 ], [ 0, 1, 0, 0 ], [ 1, 0, 0, 0 ] ]
\endexample


\>TransformationOnLevel( <a>, <lev> ) O
\>TransformationOnFirstLevel( <a> ) O

The first function returns the transformation induced by the tree homomorphism
<a> on the level <lev>. See also `PermOnLevel'~("PermOnLevel").

If the transformation is invertible then it returns a permutation, and
`Transformation' otherwise.

`TransformationOnFirstLevel'(<a>) is equivalent to
`TransformationOnLevel'(<a>, `1').


\>TransformationOnLevelAsMatrix( <g>, <lev> ) F

Computes the action of the element <g> on the <lev>-th level as a permutational matrix, in
which the i-th row contains 1 at the position i^<g>.
\beginexample
gap> L := AutomatonSemigroup("p=(p,q)(1,2), q=(p,q)[1,1]");
< p, q >
gap> TransformationOnLevel(p*q,2);
Transformation( [ 1, 1, 2, 2 ] )
gap> TransformationOnLevelAsMatrix(p*q,2);
[ [ 1, 0, 0, 0 ], [ 1, 0, 0, 0 ], [ 0, 1, 0, 0 ], [ 0, 1, 0, 0 ] ]
\endexample


\>Word( <a> ) O

Returns <a> as an associative word (an element of the underlying free group) in
the generators of the self-similar group (semigroup) to which <a> belongs.
\beginexample
gap> L := AutomatonGroup("p=(p,q)(1,2), q=(p,q)");
< p, q >
gap> w := Word(p*q^2*p^-1);
p*q^2*p^-1
gap> Length(w);
4
\endexample




%%%%%%%%%%%%%%%%%%%%%%%%%%%%%%%%%%%%%%%%%%%%%%%%%%%%%%%%%%%%%%%%%%
\Section{Operations with tree automorphisms and homomorphisms}

The multiplication of tree homomorphisms is defined in the standard way 

\>`<a> \* <b>'{product}!{for tree homomorphisms}

The following operations allow computation of the actions of tree 
homomorphisms on letters and vertices

\>`<letter> ^ <a>'{action}!{of tree homomorphism on letter}
\>`<vertex> ^ <a>'{action}!{of tree homomorphism on vertex}

\beginexample
gap> L := AutomatonGroup("p=(p,q)(1,2), q=(p,q)");                             
< p, q >
gap> 1^p;
2
gap> [1, 2, 2, 1, 2, 1]^(p*q^2);
[ 2, 1, 2, 2, 1, 2 ]
\endexample


The operations below describe how to work with sections of tree homomorphisms.

\>Section( <a>, <v> ) O

Returns the section of the automorphism (homomorphism) <a> at the vertex <v>.
The vertex <v> can be a list representing the vertex, or a positive integer
representing a vertex of the first level of the tree.
\beginexample
gap> L := AutomatonGroup("p=(p,q)(1,2), q=(p,q)");
< p, q >
gap> Section(p*q*p^2, [1,2,2,1,2,1]);
p^2*q^2
\endexample


\>Sections( <a> [, <lev>] ) O

Returns the list of sections of <a> at the <lev>-th level. If <lev> is omitted
it is assumed to be 1.
\beginexample
gap> L := AutomatonGroup("p=(p,q)(1,2), q=(p,q)");
< p, q >
gap> Sections(p*q*p^2);
[ p*q^2*p, q*p^2*q ]
\endexample


\>Decompose( <a>[, <k>] ) O

Returns the decomposition of the tree homomorphism <a> on the <k>-th level of the tree, i.e. the
representation of the form $$a = (a_1, a_2, \ldots, a_{d_1\times...\times d_k})\sigma$$
where $a_i$ are the sections of <a> at the <k>-th level, and $\sigma$ is the
transformation of the <k>-th level. If <k> is omitted it is assumed to be 1.
\beginexample
gap> L := AutomatonGroup("p=(p,q)(1,2), q=(p,q)");
< p, q >
gap> Decompose(p*q^2);
(p*q^2, q*p^2)(1,2)
gap> Decompose(p*q^2,3);
(p*q^2, q*p^2, p^2*q, q^2*p, p*q*p, q*p*q, p^3, q^3)(1,8,3,5)(2,7,4,6)
\endexample







\>`<a> in <G>'{in}

Returns whether the automorphism <a> belongs to the group <G>. In some cases it does not stop.
\beginexample
gap> L := AutomatonGroup("p=(p,q)(1,2), q=(p,q)");                             
< p, q >
gap> H := Group([p^2, q^2]);
< p^2, q^2 >
gap> p in H;
false
\endexample



\>OrbitOfVertex( <ver>, <g>[, <n>] ) O

Returns the list of vertices in the orbit of the vertex <ver> under the
action of the semigroup generated by the automorphism <g>.
If <n> is specified, it returns only the first <n> elements of the orbit.
Vertices are defined either as lists with entries from $\{1,\ldots,d\}$, or as
strings containing characters $1,\ldots,d$, where $d$
is the degree of the tree.
\beginexample
gap> T := AutomatonGroup("t=(1,t)(1,2)");
< t >
gap> OrbitOfVertex([1,1,1], t);
[ [ 1, 1, 1 ], [ 2, 1, 1 ], [ 1, 2, 1 ], [ 2, 2, 1 ], [ 1, 1, 2 ],
  [ 2, 1, 2 ], [ 1, 2, 2 ], [ 2, 2, 2 ] ]
gap> OrbitOfVertex("11111111111", t, 6);
[ [ 1, 1, 1, 1, 1, 1, 1, 1, 1, 1, 1 ], [ 2, 1, 1, 1, 1, 1, 1, 1, 1, 1, 1 ],
  [ 1, 2, 1, 1, 1, 1, 1, 1, 1, 1, 1 ], [ 2, 2, 1, 1, 1, 1, 1, 1, 1, 1, 1 ],
  [ 1, 1, 2, 1, 1, 1, 1, 1, 1, 1, 1 ], [ 2, 1, 2, 1, 1, 1, 1, 1, 1, 1, 1 ] ]
\endexample


\>PrintOrbitOfVertex( <ver>, <g>[, <n>] ) O

Prints the orbit of the vertex <ver> under the action of the semigroup generated by
<g>. Each vertex is printed as a string containing characters $1,\ldots,d$, where $d$
is the degree of the tree. In case of binary tree the symbols `` '' and ```x'''
are used to represent `1' and `2'.
If <n> is specified only the first <n> elements of the orbit are printed.
Vertices are defined either as lists with entries from $\{1,\ldots,d\}$, or as
strings. See also `OrbitOfVertex' ("OrbitOfVertex").
\beginexample
gap> L := AutomatonGroup("p=(p,q)(1,2), q=(p,q)");
< p, q >
gap> PrintOrbitOfVertex("2222222222222222222222222222222", p*q^-2, 6);
xxxxxxxxxxxxxxxxxxxxxxxxxxxxxxx
 x x x x x x x x x x x x x x x
x  xx  xx  xx  xx  xx  xx  xx
   x   x   x   x   x   x   x
xxx    xxxx    xxxx    xxxx
 x     x x     x x     x x
gap> H := AutomatonGroup("t=(s,1,1)(1,2,3), s=(t,s,t)(1,2)");
< t, s >
gap> PrintOrbitOfVertex([1,2,1], s^2);
121
132
123
131
122
133
\endexample


\>PermActionOnLevel( <perm>, <big_lev>, <sm_lev>, <deg> ) F

Given a permutation <perm> on the <big_lev>-th level of the tree of degree
<deg> returns the permutation induced by <perm> on a smaller level
<sm_lev>.
\beginexample
gap> PermActionOnLevel((1,4,2,3), 2, 1, 2);
(1,2)
gap> PermActionOnLevel((1,13,5,9,3,15,7,11)(2,14,6,10,4,16,8,12), 4, 2, 2);
(1,4,2,3)
\endexample




%%%%%%%%%%%%%%%%%%%%%%%%%%%%%%%%%%%%%%%%%%%%%%%%%%%%%%%%%%%%%%%%%%
\Section{Elements of groups and semigroups defined by wreath recursion}

\>`IsFiniteState( <a> )'{IsFiniteState![selfsim]}@{`IsFiniteState'!`[selfsim]'} P

Returns `true' if <a> has finitely many different sections.
It will never stop if the free reduction of words is not sufficient
to establish the finite-state property or if <a> is not finite-state (has
infinitely many different sections).

See also `AllSections' ("AllSections") for the list of all sections and
`MealyAutomaton' ("MealyAutomaton"), which allows to construct
a Mealy automaton whose states are the sections of <a> and which
encodes its action on the tree.
\beginexample
gap> D := SelfSimilarGroup("x=(1,y)(1,2), y=(z^-1,1)(1,2), z=(1,x*y)");
< x, y, z >
gap> IsFiniteState(x*y^-1);
true
\endexample


\>AllSections( <a> ) A

Returns the list of all sections of <a> if there are finitely many of them and
this fact can be established using free reduction of words in sections. Otherwise
will never stop. Note, that it does not check whether all elements of the list
are actually different automorphisms (homomorphisms) of the tree.
\beginexample
gap> D := SelfSimilarGroup("x=(1,y)(1,2), y=(z^-1,1)(1,2), z=(1,x*y)");
< x, y, z >
gap> AllSections(x*y^-1);
[ x*y^-1, z, 1, x*y, y*z^-1, z^-1*y^-1*x^-1, y^-1*x^-1*z*y^-1, z*y^-1*x*y*z,
  y*z^-1*x*y, z^-1*y^-1*x^-1*y*z^-1, x*y*z, y, z^-1, y^-1*x^-1, z*y^-1 ]
\endexample



See also operation `MealyAutomaton' ("MealyAutomaton"), which allows to construct 
a Mealy automaton whose states are the sections of given tree homomorphism and which
encodes its action on the tree.


%%%%%%%%%%%%%%%%%%%%%%%%%%%%%%%%%%%%%%%%%%%%%%%%%%%%%%%%%%%%%%%%%%
\Section{Elements of contracting groups}

\>AutomPortrait( <a> ) F
\>AutomPortraitBoundary( <a> ) F
\>AutomPortraitDepth( <a> ) F

Constructs the portrait of an element <a> of a
contracting group $G$. The portrait of <a> is defined recursively as follows.
For $g$ in the nucleus of $G$ the portrait is just $[g]$. For any other
element $g=(g_1,g_2,\ldots,g_d)\sigma$ the portrait of $g$ is
$[\sigma, `AutomPortrait'(g_1),\ldots, `AutomPortrait'(g_d)]$, where $d$ is
the degree of the tree. This structure describes a finite tree whose inner vertices
are labelled by permutations from $S_d$ and the leaves are labelled by
elements from the nucleus. The contraction in $G$ guarantees that the
portrait of any element is finite.

The portraits may be considered as ``normal forms''
of the elements of $G$, since different elements have different portraits.

One also can be interested only in the boundary of a portrait, which consists
of all leaves of the portrait. This boundary can be described by an ordered set of
pairs $[level_i, g_i]$, $i=1,\ldots,r$ representing the leaves of the tree ordered from left
to right (where $level_i$ and $g_i$ are the level and the label of the $i$-th leaf
correspondingly, $r$ is the number of leaves). The operation `AutomPortraitBoundary'(<a>)
computes this boundary.

`AutomPortraitDepth'( <a> ) returns the depth of the portrait, i.e. the minimal
level such that all sections of <a> at this level belong to the nucleus of $G$.

\beginexample
gap> Basilica := AutomatonGroup("u=(v,1)(1,2), v=(u,1)");
< u, v >
gap> AutomPortrait(u^3*v^-2*u);
[ (), [ (), [ (), [ v ], [ v ] ], [ 1 ] ],
  [ (), [ (), [ v ], [ u^-1*v ] ], [ v^-1 ] ] ]
gap> AutomPortrait(u^3*v^-2*u^3);
[ (), [ (), [ (1,2), [ (), [ (), [ v ], [ v ] ], [ 1 ] ], [ v ] ], [ 1 ] ],
  [ (), [ (1,2), [ (), [ (), [ v ], [ v ] ], [ 1 ] ], [ u^-1*v ] ], [ v^-1 ]
     ] ]
gap> AutomPortraitBoundary(u^3*v^-2*u^3);
[ [ 5, v ], [ 5, v ], [ 4, 1 ], [ 3, v ], [ 2, 1 ], [ 5, v ], [ 5, v ],
  [ 4, 1 ], [ 3, u^-1*v ], [ 2, v^-1 ] ]
gap> AutomPortraitDepth(u^3*v^-2*u^3);
5
\endexample




%%%%%%%%%%%%%%%%%%%%%%%%%%%%%%%%%%%%%%%%%%%%%%%%%%%%%%%%%%%%%%%%%%
