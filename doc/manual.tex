<?xml version="1.0" encoding="ISO-8859-1"?>

<!DOCTYPE Book SYSTEM "gapdoc.dtd"
 [ <!ENTITY see '<Alt Only="LaTeX">$\to$</Alt><Alt Not="LaTeX">--&tgt;</Alt>'>]
 [ <!ENTITY autgps '<Verb>AutomataGroups</Verb>'>]>

<Book Name="Automata groups">

<TitlePage>
  <Title>AutomataGroups</Title>
  <Subtitle>Package for computations in groups generated by finite automata</Subtitle>
  <Version>Version 0.1</Version>
  <Author>Yevgen Muntyan
          <Email>muntyan@math.tamu.edu</Email>
  </Author>
  <Author>Dmytro Savchuk
          <Email>savchuk@math.tamu.edu</Email>
          <Homepage>http://www.math.tamu.edu/~savchuk</Homepage>
  </Author>
  <Date>December 2006</Date>
  <Address>
  Department of Mathematics<Br/> Texas A&tamp;M University<Br/> College Station, 77843<Br/> TX, USA
  </Address>
  <Abstract>This is a manual for an &autgps; package, implementing in &GAP; basic functions and algorithms
  for groups generated by finite automata, contracting groups.
  </Abstract>
  <Copyright>&copyright; 2006 by Yevgen Muntyan and Dmytro Savchuk
  </Copyright>
  <Acknowledgements>The development of this package was partially supported by National Science Foundation
  </Acknowledgements>
  <Colophon>The project was originally started in 2000 mostly for personal use. It was gradually
    expadnig during consequent years, including both addition of new algorithms and simplification of
    user interface. It was used in the process of classification of groups generated by $3$-state
    automata over 2-letter alphabet.
  </Colophon>
</TitlePage>

<TableOfContents/>

<Body>
<Chapter Label="ch:intro"><Heading>Introduction</Heading>
 <Section Label="sec:math_background"><Heading>Mathematical Background</Heading>

 </Section>

 <Section><Heading>Notations and Agreements</Heading>

 </Section>

 <Section><Heading>Quick Example</Heading>

 </Section>
</Chapter>

<Chapter Label="ch:automgrps"><Heading>Automata groups</Heading>
 <Section><Heading>Trees</Heading>
 These functions allow to construct and operate with vertices of the trees.

 <#Include Label="VertexNumber">

 <#Include Label="NumberOfVertex">
 </Section>

 <Section><Heading>Tree Automorphisms</Heading>
 These functions allow to construct orbit of vertex.

 <#Include Label="OrbitOfVertex">

 <#Include Label="PrintOrbitOfVertex">
 </Section>

</Chapter>


<Chapter Label="ch:contr"><Heading>Contracting groups</Heading>

 <#Include Label="ContractingLevel">
</Chapter>

</Body>


<!--<Bibliography Databases="example" Style="alpha"/>-->
<TheIndex/>

</Book>

