% This file was created automatically from autom.msk.
% DO NOT EDIT!
%%%%%%%%%%%%%%%%%%%%%%%%%%%%%%%%%%%%%%%%%%%%%%%%%%%%%%%%%%%%%%%%%%
\Chapter{Noninitial Automata}

%%%%%%%%%%%%%%%%%%%%%%%%%%%%%%%%%%%%%%%%%%%%%%%%%%%%%%%%%%%%%%%%%%
\Section{Definition}

\>Automaton( <table>[, <names>[, <alphabet>]] ) O
\>Automaton( <string> ) O

Creates a Mealy automaton defined by the <table> or <string>. Format of the <table> is
the following: it is a list of states, where each state is a list of
positive integers which represent transition function at given state and a
permutation or transformation which represent output function at this
state.  Format of string <string> is the same as in `AutomGroup' (see~"AutomGroup").

\beginexample
A:=Automaton([[1,2,(1,2)],[3,1,()],[3,3,(1,2)]],["a","b","c"]);
<automaton>
gap> Print(A);
a = (a, b)(1,2), b = (c, a), c = (c, c)(1,2)
gap> B:=Automaton([[1,2,Transformation([1,1])],[3,1,()],[3,3,(1,2)]],["a","b","c"]);
<automaton>
gap> Print(B);
a = (a, b)[ 1, 1 ], b = (c, a), c = (c, c)[ 2, 1 ]
D:=Automaton("a=(a,b)(1,2),b=(b,a)");
<automaton>
\endexample


\>IsAutomaton( <A> ) C

A category of non-initial finite Mealy automata with the same input and
output alphabet.


\>NumberOfStates( <A> ) A

Returns the number of states of automaton <A>.



\>SizeOfAlphabet( <A> ) A

Returns the number of letters in the alphabet automaton <A> acts on.



\>`AutomatonList( <A> )'{AutomatonList![automaton]}@{`AutomatonList'!`[automaton]'} A

Returns a list of <A> acceptible by `Automaton' (see "Automaton")





%%%%%%%%%%%%%%%%%%%%%%%%%%%%%%%%%%%%%%%%%%%%%%%%%%%%%%%%%%%%%%%%%%
\Section{Tools}

\>IsTrivial( <A> ) O

Computes whether or not automaton <A> is equivalent to the trivial automaton.
\beginexample
gap> A:=Automaton("a=(c,c),b=(a,b),c=(b,a)");
<automaton>
gap> IsTrivial(A);
true
\endexample


\>IsInvertible( <A> ) P

Is `true' if <A> is invertible and `false' otherwise.


\>MinimizationOfAutomaton( <A> ) F

Returns an automaton obtained from automaton <A> by minimization.
\beginexample
gap> B:=Automaton("a=(1,a)(1,2),b=(1,a)(1,2),c=(a,b),d=(a,b)");
<automaton>
gap> C:=MinimizationOfAutomaton(B);
<automaton>
gap> Print(C);
a = (1, a)(1,2), c = (a, a), 1 = (1, 1)
\endexample


\>MinimizationOfAutomatonTrack( <A> ) F

Returns the list `[A_new,track_s,track_l]', where `A_new' is an
automaton obtained from automaton <A> by minimization,
`track_s' is how new states are expressed in terms of the old ones, and
`track_l' is how old states are expressed in terms of the new ones.
\beginexample
gap> B:=Automaton("a=(1,a)(1,2),b=(1,a)(1,2),c=(a,b),d=(a,b)");
<automaton>
gap> B_min:=MinimizationOfAutomatonTrack(B);
[ <automaton>, [ 1, 3, 5 ], [ 1, 1, 2, 2, 3 ] ]
gap> Print(B_min);
[ a = (1, a)(1,2), c = (a, a), 1 = (1, 1), [ 1, 3, 5 ], [ 1, 1, 2, 2, 3 ] ]
\endexample


\>IsOfPolynomialGrowth( <A> ) P

Determines whether an automaton <A> has polynomial growth in terms of Sidki~\cite{sidki:circuit}.

See also `IsBounded' ("IsBounded" and
`PolynomialDegreeOfGrowthOfAutomaton' ("PolynomialDegreeOfGrowthOfAutomaton").
\beginexample
gap> B:=Automaton("a=(b,1)(1,2),b=(a,1)");
<automaton>
gap> IsOfPolynomialGrowth(B);
true
gap> D:=Automaton("a=(a,b)(1,2),b=(b,a)");
<automaton>
gap> IsOfPolynomialGrowth(D);
false
\endexample


\>IsBounded( <A> ) P

Determines whether an automaton <A> is bounded in terms of Sidki~\cite{sidki:circuit}.

See also `IsOfPolynomialGrowth' ("IsOfPolynomialGrowth")
and `PolynomialDegreeOfGrowthOfAutomaton' ("PolynomialDegreeOfGrowthOfAutomaton").
\beginexample
gap> B:=Automaton("a=(b,1)(1,2),b=(a,1)");
<automaton>
gap> IsBounded(B);
true
gap> C:=Automaton("a=(a,b)(1,2),b=(b,c),c=(c,1)(1,2)");
<automaton>
gap> IsBounded(C);
false
\endexample


\>`PolynomialDegreeOfGrowthOfAutomaton( <A> )'{PolynomialDegreeOfGrowthOfAutomaton![automaton]}@{`PolynomialDegreeOfGrowthOfAutomaton'!`[automaton]'} A

For an automaton <A> of polynomial growth in terms of Sidki~\cite{sidki:acyclic}
determines its degree of
polynomial growth. This degree is 0 if and only if automaton is bounded.
If the growth of automaton is exponential returns `fail'.

See also `IsOfPolynomialGrowth' ("IsOfPolynomialGrowth")
and `IsBounded' ("IsBounded").
\beginexample
gap> B:=Automaton("a=(b,1)(1,2),b=(a,1)");
<automaton>
gap> PolynomialDegreeOfGrowthOfAutomaton(B);
0
gap> C:=Automaton("a=(a,b)(1,2),b=(b,c),c=(c,1)(1,2)");
<automaton>
gap> PolynomialDegreeOfGrowthOfAutomaton(C);
2
\endexample


\>DualAutomaton( <A> ) O

Returns an automaton dual to <A>.
\beginexample
gap> A:=Automaton("a=(b,a)(1,2),b=(b,a)");
<automaton>
gap> D:=DualAutomaton(A);
<automaton>
gap> Print(D);
d1 = (d2, d1)[ 2, 2 ], d2 = (d1, d2)[ 1, 1 ]
\endexample


\>InverseAutomaton( <A> ) O

Returns an automaton inverse to <A> if <A> is invertible.
\beginexample
gap> A:=Automaton("a=(b,a)(1,2),b=(b,a)");
<automaton>
gap> B:=InverseAutomaton(A);
<automaton>
gap> Print(B);
a1 = (a1, a2)(1,2), a2 = (a2, a1)
\endexample


\>IsBireversible( <A> ) O

Computes whether or not automaton <A> is bireversible, i.e. <A>, dual to <A> and
dual to the inverse of <A> are invertible. The example below shows that the
Bellaterra automaton is bireversible.
\beginexample
gap> Bellaterra:=Automaton("a=(c,c)(1,2),b=(a,b),c=(b,a)");
<automaton>
gap> IsBireversible(Bellaterra);
true
\endexample


\>DisjointUnion( <A>, <B> ) O

Coonstructs a disjoint union of automata <A> and <B>
\beginexample
gap> A:=Automaton("a=(a,b)(1,2),b=(a,b)");
<automaton>
gap> B:=Automaton("c=(d,c),d=(c,e)(1,2),e=(e,d)");
<automaton>
gap> Print(DisjointUnion(A,B));
a1 = (a1, a2)(1,2), a2 = (a1, a2), a3 = (a4, a3), a4 = (a3, a5)(1,2), a5 = (a5, a4)
a1 = (a1, a2)(1,2), a2 = (a1, a2), a3 = (a4, a3), a4 = (a3, a5)
(1,2), a5 = (a5, a4)
\endexample


\>`<A> \* <B>'{product}!{for noninitial automata}

Constructs a product of 2 noninitial automata <A> and <B>.
\beginexample
gap> A:=Automaton("a=(a,b)(1,2),b=(a,b)");        
<automaton>
gap> B:=Automaton("c=(d,c),d=(c,e)(1,2),e=(e,d)");
<automaton>
gap> Print(A*B);                                  
a1 = (a1, a5)(1,2), a2 = (a3, a4), a3 = (a2, a6)
(1,2), a4 = (a2, a4), a5 = (a1, a6)(1,2), a6 = (a3, a5)
\endexample

\>IsEquivAutomata( <A>, <B> ) O

Returns `true' if for every state `s' of automaton <A> there is a state of automaton <B>
equivalent to `s' and vice versa.
\beginexample
gap> A:=Automaton("a=(b,a)(1,2),b=(a,c)(),c=(b,c)(1,2)");
<automaton>
gap> B:=Automaton("b=(a,c)(),c=(b,c)(1,2),a=(b,a)(1,2),d=(b,c)(1,2)");
<automaton>
gap> IsEquivAutomata(A,B);
true
\endexample





%%%%%%%%%%%%%%%%%%%%%%%%%%%%%%%%%%%%%%%%%%%%%%%%%%%%%%%%%%%%%%%%%%
